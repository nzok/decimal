\documentclass{article}
\usepackage{listings}
\lstloadlanguages{Haskell}
\lstnewenvironment{code}
    {\lstset{}%
      \csname lst@SetFirstLabel\endcsname}
    {\csname lst@SaveFirstLabel\endcsname}
    \lstset{
      basicstyle=\small\ttfamily,
      flexiblecolumns=false,
      basewidth={0.5em,0.45em},
      literate={+}{{$+$}}1 {/}{{$/$}}1 {*}{{$*$}}1 {=}{{$=$}}1
               {>}{{$>$}}1 {<}{{$<$}}1 {\\}{{$\lambda$}}1
               {\\\\}{{\char`\\\char`\\}}1
               {->}{{$\rightarrow$}}2 {>=}{{$\geq$}}2 {<-}{{$\leftarrow$}}2
               {<=}{{$\leq$}}2 {=>}{{$\Rightarrow$}}2
               {\ .}{{$\circ$}}2 {\ .\ }{{$\circ$}}2
               {>>}{{>>}}2 {>>=}{{>>=}}2
               {|}{{$\mid$}}1
    }
\begin{document}

\section{Testing}

The previous section showed that a rigorous definition
of fixed-point arithmetic suitable for use in writing
standards can be written.  That is not surprising.  The
claim of this paper is that machine-checkable specifications
{\it should} be written for appropriate aspects of standards,
because machine checking is {\it needed}.

We can classify errors in a specification as
\begin{itemize}
\item {\it Syntax} errors, such as incorrect punctuation,
easily caught by a parser.  Reducing these calls for a
notation that has a checkable grammar.
\item {\it Type} errors, such as calling the wrong function,
or providing the wrong number of arguments, or swapping two
arguments, caught by a more or less capable type checker.
Reducing these calls for a notation that has a statically
checkable type system.
\item {\it Avoidable} semantic errors, such as swapping
two arguments of the same type, which can be turned into
type errors by better use of the type system.  For
example, Scale was introduced to make numeric values and
scales into different types.
\item Other {\it Semantic} errors, such as swapping the
arguments of subtraction (which must be of the same type,
so the error cannot be avoided), or sign errors.
Reducing these requires a notation that can be executed
in order to run tests, and ideally a testing framework
that makes it easy to write tests.
\end{itemize}

The tests for the `convert' function reported errors in
the Down and Up rounding modes, which were errors in the
test code, and errors in the Even and Odd rounding modes.
The error there was that if the argument was already an
(odd, even) integer, it was returned instead of an
(even, odd) result.  This revealed that the informal
specification of the Even and Odd rounding modes was
ambiguous.

\end{document}
