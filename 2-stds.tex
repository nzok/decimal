\documentclass{article}

\begin{document}

\def\subI{${}_I$}
\def\subF{${}_F$}
\def\subD{${}_D$}
\def\ifso{\mathbin{?}}

\section{Rounding}

Converting a rational number $x \in {\cal Q}$ to a nearby
integer $n \in {\cal Z}$ can be done in surprisingly many
ways.  We shall take the floor operation
\[ \lfloor x \rfloor = \max \{n \in {\cal Z} | n \le x\} \]
as the basic operation.  We also need $\bot$ (undefined),
even$(n)$ testing whether $n$ is a multiple of 2,
$p \ifso t : f$ (if $p$ then $t$ else $f$), and $\sigma(x)$ =
$x\ge 0 \ifso 1 : -1$.

\begin{tabular}{lll}
exact & $x=\lfloor x\rfloor \ifso x : \bot$ & exception if not integral\\
down  & $\lfloor x \rfloor$             & floor\\
up    & $\lceil x \rceil = -\lfloor -x \rfloor$ & ceiling\\
in    & $\sigma(x)\times\lfloor|x|\rfloor$ & truncate\\
out   & $\sigma(x)\times\lfloor|x|+\frac12\rfloor$ & away from zero\\
evenr & $\mbox{even}(\lfloor x\rfloor) \ifso \lfloor x \rfloor
        : \lceil x \rceil$ & round to even\\
oddr  & $\mbox{even}(\lfloor x\rfloor) \ifso \lceil x \rceil
        : \lfloor x \rfloor$ & round to odd (not used in practice)
\end{tabular}

None of these is rounding.  For that we need
\[\mbox{nearest} f x = d > \frac12 \ifso \lceil x \rceil :
 d < \frac12 \ifso \lfloor x \rfloor : f(x) \]
\noindent where
\[ d = x - \lfloor x \rfloor \]
\
This higher-order function can be combined with the seven conversions
above to make seven ``round to nearest'' modes such as nearest out
(banker's rounding) or nearest evenr (IEEE rounding).

\section{Existing standards}

\subsection{ISO 10967 Language-independent arithmetic}

Roughly speaking, the three parts of LIA cover everything
in Fortran 95 and C99 and a little bit more.

We have to discuss ISO 10967 for two reasons: it is the
standard that is meant to serve as a reference for arithmetic
in programming languages, and it does define fixed point
numbers (but no operations on them).  Any definition of
fixed point arithmetic should be practically compatible with it.

Further, the 1998 edition provided axioms (though no evidence
that the system described was consistent with them) which could
serve as models for checking a fixed-point specification.

The Language-independent arithmetic standard has three parts.

\subsubsection{Part: 1 Integer and floating point arithmetic}
An integer data type $I$ has parameters {\it bounded}\subI{}
true if the set is finite, {\it minint}\subI{} being the
smallest value or $-\infty$, and {\it maxint}\subI{} being the
largest value or $\infty$.  If {\it bounded}\subI{} is true,
then {\it minint}\subI{} can be 0, -{\it maxint}\subI{}, or
-({\it maxint}\subI{}+1).  Curiously, while both signed and
unsigned bounded integers are allowed, unbounded integers
must be signed; ``unbounded natural numbers are not covered
by this document.''  The usual 6 comparison operations are
defined.

The arithmetic operations are {\it neg}\subI{} (unary negation),
{\it add}\subI{}, {\it sub}\subI{}, {\it mul}\subI{},
flooring division {\it quot}\subI{} and the corresponding
remainder {\it mod}\subI{}, {\it abs}\subI{},
and {\it signum}\subI{} taking values 1 and -1.  The 1998 edition
included {\it sign}\subI{} with values 1, 0, and -1, but that was
dropped.  The 1998 edition included
truncating division {\it div}${}^t_I$ and the corresponding
remainder {\it rem}${}^t_I$ and two additional remainder functions
{\it mod}${}^a_I$ and {\it mod}${}^p_I$, all of which were deleted
in the 2012 edition.

LIA-1 integer arithmetic must handle overflow by causing
an overflow indication.  This includes conversion to a 
smaller integer type.  Wrapping arithmetic is covered in
part 2.

A floating point data type $F$ has parameters {\it r}\subF{}
(radix), {\it p}\subF{} (precision), {\it emax}\subF{} (largest
exponent), {\it emin}\subF{} (smallest exponent),
{\it denorm}\subF{} (whether IEEE-style subnormals are supported),
and {\it ice\_60559}\subF{} (whether $F$ conforms to the IEEE
standard).  The usual 6 comparison operations are defined,
taking care to dodge around infinities and NaNs.

Additional tests {\it isnegzero}\subF{}, {\it istiny}\subF{},
{\it isnan}\subF{}, and {\it issignan}\subF{} express tests that
cannot be done using ordinary comparison against zero.

Three sets of basic operations (rounding to nearest even, down,
or up) are defined if {\it denorm}\subF{} is true, just the
rounded to nearest set if {\it denorm}\subF{} is false.
The basic operations are {\it add}, {\it sub}, {\it mul},
{\it div}, and {\it sqrt}.

Some arithmetic operations are not sensitive to rounding
mode: {\it abs}\subF{}, two-valued {\it signum}\subF{},
the IEEE remainder {\it residue}\subF{}, {\it intpart}\subF{},
{\it fractpart}\subF{}, {\it succ}\subF{} (next higher),
{\it pred}\subF{} (next lower), and {\it ulp}\subF{}.

There are three conversions from floating point to integer:
{\it ceiling}${}_{F\rightarrow I}$,
{\it floor}${}_{F\rightarrow I}$, and
{\it rounding}${}_{F\rightarrow I}$.

Section 5.3.5 defines floating point to fixed point conversion
(for the sake of format conversions like C's {\tt \%.$n$f}) and in
passing defines unbounded fixed point types, with parameters
{\it r}\subD{} (radix), {\it d}\subD{} (``density'' = scale),
and {\it dmax}\subD{} (a positive element of $\cal R$ or
undefined).  Two sets are defined:
\begin{eqnarray*}
D^* &=& \{n/(r_D^{d_D}) | n \in {\cal Z}\}\\
D &=& D^* \mbox{if unbounded}\\
D &=& D^* \cap [-dmax_D,dmax_D] \mbox{if bounded}
\end{eqnarray*}
Three rounding functions {\it nearest}\subD{} (our nearest evenr),
{\it up}\subD{} (ceiling), and
{\it down}\subD{} (floor), all ${\cal R} \rightarrow D^*$, are
defined.  Conversion from floating point to fixed point and
from fixed point to floating point uses one of these rounding
modes just like the basic arithmetic operations.

However, fixed point exists in LIA {\it only} for I/O conversion;
there are no comparison or arithmetic operations on $D$.

Floating point arithmetic with truncation is conspiciously
absent from the 2012 edition of LIA-1.  What it describes is
accurately rounded arithmetic and the building blocks for
interval arithmetic.

\subsubsection{Part: 2 Elementary numerical functions}

This part was issued in 2001.  There are some slight ``continuity
errors'' due to the revision of LIA-1: {\it iec\_559}\subF{} is
now {\it iec\_60559}\subF{}, for example.  This is another example
of why standards need to be machine-checked.

The additional integer operations include maximum, minimum,
greatest common divisor, and least common multiple
of a pair or sequence, if $x\ge y$ then $x-y$ else 0 (diminish),
integer power (but broken for $0^0$ so of very little use),
floored shifting by a power of 2 or 10, floored square root,
divisibility, evenness, oddness, floored ({\it quot}\subI{},
{\it mod}\subI{}), rounded ({\it ratio}\subI{}, {\it residue}\subI{}),
and ceilinged ({\it group}\subI{}, {\it pad}) quotient and remainder.
(The {\it pad} operation is actually $\lceil x/y \rceil\times y - x$.)

There are also wrapping versions of {\it add}, {\it sub}, and
{\it mul}, plus ``overflow'' versions giving you the high bits
lost by wrapping so that multiprecision arithmetic can be written.

The additional floating point operations include maximum and
minimum of a pair or sequence, diminish,  conversion to integral
in floating form ({\it floor}\subF{}, {\it rounding}\subF{}, and
{\it ceiling}\subF{}) together with {\it *\_rest}\subF{} operations
providing the lost fraction parts exactly, square root (already
in part 1) and reciprocal square root.  There is an operation
for multiplying two floats giving a result in a wider format with
a single rounding, just right for dot product.

The elementary transcendental operations include raising a
floating point number to an integer or floating point power,
plus the usual exponential, logarithmic, trigonometric, and
hyperbolic functions in several forms.

\subsubsection{Part 3: Complex integer and floating point
arithmetic and complex elementary numerical functions}

This part was issued in 2006.  It defines imaginary and
complex integer and imaginary and complex floating point
data types, basic operations on them, and elementary
transcendental operations.  Maximum and minimum are extended
to imaginary but not to complex types.

Part 3 ends with recommended changes to some of the operations
defined in Part 2 ``so that the returned angle is for the
proper quadrant''.  Again this suggests that a standard on which
great care was taken might have benefited from testing.

\section{ISO/IEC/IEEE 60559:2011}

This is the standard for floating point arithmetic.
It is the ISO/IEC edition of IEEE 754:2008, which in
turn was an update of classic IEEE 754 arithmetic to
include decimal floating point.  It specifies
32-, 64-, and 128-bit binary formats that can hold about 7,
16, and 34 decimal digits respectively
and 64- and 128-bit decimal formats that can hold
16 and 34 decimal digits respectively (table 3.2).

It is common belief that IEEE 754 arithmetic is all but
universal.  However, few systems support 128-bit binary floats
(the popular ARM, x86, and x86-64 architectures do not; SPARC
does), and even fewer support decimal floating-point natively
(recent IBM zSeries and POWER machines do).

IEEE 754 originally defined four rounding modes, up, down, in,
and nearest evenr in our terminology.  The current standard
adds nearest out (banker's rounding).

\subsection{COBOL}

COBOL is one of the oldest programming languages still in use.
It was designed to support accounting applications on small
machines.  For example, the Siemens 3003, regarded as ``a large-scale
\dots computer'', had 8000 words of
storage; the COBOL 61 compiler for that machine needed 10 passes.
Brinch Hansen and House [ref] report that ``The major problem of
implementation turned out to be the numerous definition problems
created by the vagueness of the official Cobol report.''

One of the major features of COBOL from the beginning was its
use of decimal fixed point arithmetic rather than floating point.
Accounting applications required answers that were exact or
computed according to known rounding rules.  Binary integers
and floats were eventually adopted in 2002, but had been common
extensions for many years.  

A number is characterised by (a) whether it is signed or not,
(b) its precision, and (c) its scale.  A declaration like\\
77 SOME-NUMBER PICTURE S9$(b)$V$9(a)$\\
declares a signed variable (S) with a precision of $b+a$
decimal digits and a scale of $a$.  The letter V designates
a virtual (not actually stored) decimal point.  In terms of
LIA-1, this corresponds to $r_D = 10$, $d_D = a$,
{\it dmax}${}_D = (10^{a+b}-1)/10^b$.

In LIA-1, arithmetic is only defined between values of the
same type.  In COBOL, it is extremely common for the
operands and result of an operation to be of different types.
A typical instruction is
\begin{tabbing}
www\=\kill
MULTIPLY X BY Y GIVING Z ROUNDED\\
\> ON SIZE ERROR {\it statements}\\
\>  NOT SIZE ERROR {\it statements}\\
END MULTIPLY  
\end{tabbing}
\noindent where the product $X\times Y$ is computed
according to exact rational arithmetic, rounded to the
scale of $Z$, and if the result is too large to fit,
the ON SIZE ERROR statements are executed, otherwise
the result is stored in Z and the NOT SIZE ERROR
statements are executed.

\subsubsection{Four semantics for arithmetic}

The previous paragraph is a lie.  The current COBOL standard does not
offer a semantics for arithmetic.  It offers {\it four} semantics for
arithmetic.  A COBOL program may contain an OPTIONS paragraph near
the beginning:
\begin{tabbing}
www\=wwwwwwwwwwwwwwww\=\kill
OPTIONS.\\
\> [ARITHMETIC IS \{NATIVE $|$ STANDARD $|$\\
\> \> STANDARD-BINARY $|$ STANDARD-DECIMAL \}]\\
\> [DEFAULT ROUNDED MODE IS one of eight]\\
\> \dots\\
\> [INTERMEDIATE ROUNDING IS one of four]\\
\end{tabbing}

In our terminology, the rounding modes are out, nearest out,
nearest even, nearest in, exact, up, down, and in.

There is a subtle difference between these statements:
\begin{tabbing}
MULTIPLY X BY Y GIVING Z.\\
COMPUTE Z = X * Y.
\end{tabbing}
The latter involves an arithmetic expression which has an
intermediate result that is then assigned to the destination.
The former does not.  It appears that the MULTIPLY statement
involves a single rounding whereas the COMPUTE statement may
involve a double rounding.

{\it Native} arithmetic is whatever the implementor says it is.

{\it Standard} arithmetic is what used to be standard arithmetic in
COBOL-85.  It does not cover binary or floating-point operations,
which are treated as native.  Standard arithmetic is declared to
be obsolete.  ``A standard intermediate data item is \dots the
unique value zero or an abstract, signed, normalized decimal
floating-point temporary data item \dots with a precision of
32 decimal digits'' and a scale of -999 to 999 (roughly speaking).
Intermediate values are rounded in certain circumstances.

In expressions,
addition, subtraction, multiplication, and
division are are ``the exact [result] truncated
to 32 significant digits, normalized, and stored in a standard
intermediate data item''.

{\it Standard-binary} arithmetic requires arithmetic operations
to be done according to IEEE binary floating-point
arithmetic (128-bit format).

{\it Standard-decimal} arithmetic requires arithmetic operations
to be done according to IEEE decimal floating-point arithmetic
(128-bit format).

The result is that the meaning of an arithmetic statement or
expressions can be changed by a line thousands of lines away.

The standard tries very hard to avoid using
mathematical notation.  Let us take Format 5 of the
DIVIDE statement as an example:
\begin{quote}
\begin{tabbing}
www\=\kill
DIVIDE \{identifier-2$|$literal-2\} BY \{identifier-1$|$literal-1\}\\
\> GIVINE identifier-3 [rounded-prhase]\\
\> REMAINDER identifier-4\\
\> [ON SIZE ERROR imperative-statement-1]\\
\> [NOT SIZE ERROR imperative-statement-2]\\
\> [END-DIVIDE]
\end{tabbing}

1) When native arithmetic is in effect, the quotient is the result
of dividing the dividend by the divisor.  When standard arithmetic,
stnadard-decimal arithmetic, or standard-binary arithmetic is in
effect, the quotient is the result of the arithmetic expression
$$(\mbox{dividend} / \mbox{divisor})$$
where the values of dividend and divisor are ...

6b) The quotient is then formed as specified in general rule 1
and stored in identifier-3 \dots

6c) If the size error condition is not raised, a subsidiary
quotient is developed that is signed and derived from the
quotient by truncation of digits at the least significant end
and that has the same number of digits and the same decimal point
location as the data item references by identifier-3.  The
remainder is calculated as indicated in general rules 7 and is
stored in the data item referenced by identifier-4 unless
storing the value would cause a size error condition \dots

7) When native arithmetic is in effect, the remainder is the
result of multiplying the subsidiary quotient and the divisor
and subtracting the product from the dividend.  When standard
standard arithmetic, standard-decimal arithmetic, or
standard-binary arithmetic is in effect, the remainder is the
result of the arithmetic expression
$$(\mbox{divided} - (\mbox{subsidiary-quotient} * \mbox{divisor}))$$
where \dots
\end{quote}

The reason the standard {\it does} use some expressions here is
so that the rules for intermediate rounding apply to them.  With
4 arithmetics, 8 intermediate roundings, and 4 default roundings,
a simple DIVIDE statement could mean 128 different things.

\subsection{PL/I}

PL/I was introduced in the mid-1960s as a blend of Algol,
Fortran, and COBOL.  The ECMA-50 (1976), ANSI X3.53-1976, BS 7148:1990,
NF Z 65-500, and ISO 6160:1979 standards for PL/I are all basically the
same.  They have expired.  BS ISO/IEC 6522:1992 (Information technology,
Programming languages, PL/I general purpose subset) is still current.

PL/I fixed point types are characterised by a {\it base} (2 or 10),
a {\it precision} (total number of digits), and a {\it scale} (number
of digits after the radix point).  There are actually two sets of
rules for determining the precision and scale of a result (ANSI and
IBM).  For FIXED DECIMAL they coincide"
\begin{itemize}
\item
$(p_1,s_1) \pm (p_2,s_2) \Rightarrow (p,s)$ where\\
$s = s_1 \vee s_2$\\
$p = 1+s+(p_1-s_1 \vee p_2-s_2)$
\item
$(p_1,s_1) \times (p_2,s_2) \Rightarrow (p,s)$ where\\
$s = s_1 + s_2$\\
$p = 1+p_1+p_2$
\item
$(p_1,s_1) / (p_2,s_2) \Rightarrow (p,s)$ where\\
$s = N-p_1+s_1-s+2$\\
$p = N$ and $N$ is the implementation-defined maximum precision.
\item
MOD$((p_1,s_1), (p_2,s_2)) \Rightarrow (p,s)$ where\\
$s - s_1 \vee s_2$\\
$p = N \wedge (p_2 - s_2 + s)$.
\item
$(p_1,s_1) ** k \Rightarrow (p,s)$ where\\
$s = s_1\times k$\\
$p = (p_1 + 1) \times k - 1$, if $k$ is a literal integer.
\item
ROUND$((p_1,s_1), s) \Rightarrow (p,s)$ where\\
$s$ is given as the second argument\\
$p = 1 \vee (p_1-s_1+1-s \wedge N)$
\end{itemize}
The rule for division can lead to some nasty surprises.
For example, 25+01/3 yields 25.3333333333333 but
25+1/3 results in an overflow.  So there are special functions
ADD(x, y, p[, s]), SUBTRACT(x, y, p[, s]),  MULTIPLY(x, y, p[, s]),
DIVIDE(x, y, p[, s]), 
computing a result to a specified precision and scale (default s = 0)
in a single rounding.  There is no version of MOD with precision and
scale parameters.  In assignment
statements, ``padding or truncation can occur on the left or the right.
If nonzero \dots digits on the left are lost, the SIZE condition is
raised.''  This means that a statement like\\
X = X * Y;\\
uses the ``in'' rounding mode.  Unlike COBOL, PL/I does not offer
control over rounding, not even using the special functions, so there is
no way to change this.

\subsection{SQL}
\subsection{Smalltalk}

Smalltalk-80 is the programming language that brought Object-Oriented
programming and Integrated Development ENvironments to the world.
It got an ANSI standard in 1989.  The standard includes unbounded
integers, three sizes of floating point, rational numbers, and
a ScaledDecimal class.

For the semantics of operations on ScaledDecimal numbers,
the Smalltalk standard defers to LIA.  Unfortunately, LIA has,
by intent and quite explicitly, nothing to say about fixed point
arithmetic.

The result of this is that different Smalltalk systems do quite
different things with such numbers.  For example, some Smalltalks
treat them as arbitrary rational numbers, with the scale factor
controlling only how they are printed.

\subsection{Java, C$\sharp$, and Swift}

ISO/IEC 23270:2003 defines C$\sharp$ as it was in 2002.
C$\sharp$ changes faster than ISO processes can keep up with.
The current specification is C$\sharp$ version 6.0.

Java is an industry standard pushed by Oracle (formerly by Sun);
Swift is a fast-changing language pushed by Apple.

All three languages have a type for decimal arithmetic.
In all three it is a strange hybrid between fixed point
and floating-point.  In all three the ``exponent'' range is
much smaller than the exponent range for binary floating point.

\end{document}
